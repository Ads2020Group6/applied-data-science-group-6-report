\section{Data Preparation} 
\label{sec:data-prep}

\paragraph{World Data}
The GPG is a global issue with data for many industrialised countries recorded by various organisations. For example, the Organisation for Economics Co-operation and Development (OECD) publishes gender pay gap for its 36 member countries, but it is very high level and only includes one measurement of the pay gap per year for each country. Another source of world data we explored was from UN's International Labour Organisation (ILO), with the wage gap being available as a percentage against indicators such as average hourly earnings of employees, and number of employees by gender and institutional sector.
To visualise the data, we grouped and counted them by year to see how many countries had reported their GPG and plotted the country-wise GPG. This data also only had one data-point per country per year.

\paragraph{UK GPG Data}
The UK data from the Government Equalities Office provided much more scope for investigation.
The dataset contains 25 columns including company metadata, GPG indicators and submission information, the columns used in our analysis are described in Table \ref{tab:uk-gov-data-dic} in Appendix \ref{app:data-dic}.  The calculation methods for the required metrics are unambiguously defined on the \texttt{gov.uk} website \cite{govuk-how-to-calculate}. However, these figures are self-reported, and not audited independently. Data from 2017-18 and 2018-19 years include records of almost 11,000 companies. Owing to the COVID-19 pandemic, the requirement for 2019 submissions has been suspended indefinitely, resulting in a partial dataset. At the time of writing, only 5463 companies had reported their pay gap for 2019. General and domain specific transformation techniques are described in the following paragraphs; their effectiveness for modelling is discussed in Section \ref{sec:modelling}.

\subsection{Data acquisition}
Our data ingress module automatically retrieves and stores data from 2017, 2018 and 2019 UK GPG reports from the \code{uk.gov} website. Additionally, it fetches a dataset of Standard Industrial Classification (SIC) data from a public Github repository, which is used during data augmentation to expand details about industry sections the companies belong to (see section \ref{sic-agment}). Our data collector module avoids downloading existing data unless explicitly requested. This design resulted helpful since the companies are still submitting their 2019 reports: we observed up to 15 new entries in the days prior to this report's release.

\subsection{Cleaning and transforming the UK GPG data}
\paragraph{Data Removal} Our cleaning pipeline starts by removing duplicates, columns whose values are unique in all rows and unused (e.g. \code{CompanyName}), as well as irrelevant ones such as \code{DateSubmitted}. Although we explored \code{Address} data, we found that these indicate headquarters and make for invalid geo-related conclusions about regional pay gap conditions, so the column was subsequently dropped.

\paragraph{Data Imputation} All of the target variables (\code{DiffHourlyPercent} and \code{DiffBonusPercent} for mean and median) had missing values, which were imputed with the means and medians of their respective columns in order to follow the statistic description they represent.

\paragraph{Categorical Data} 
\code{EmployerSize} contains seven values, six of which indicate a category of company size by number of employees, and the other being `Not Specified'. We dropped records without this information, because they constituted less than 1\% of the total. The company size distribution for 2018 is shown in Figure\ \ref{fig:employer-size-dist}.
We transformed these features using three techniques: midpoint quantification, one hot encoding and indexing. Firstly, we created a new column with midpoint values of the the given range. For example, when the original value is \code{500 to 999} or \code{20,000 and more} we augmented these rows with 750 and 35,000, respectively. For one-hot encoding\ \cite{pandas-get-dummies}, we created 6 new columns, one for each possible employer size. The respective size column is set to \code{1}, and the others to \code{0}. The third approach was to use label encoding where each category gets a numeric label starting from zero. We found the first two useful in our modelling experiments (Section\ \ref{sec:modelling}).

\paragraph{Transforming Industry Sectors}
\label{sic-agment}
The \code{SicCodes} column contains one or multiple SIC codes (Standard Industrial Classification of Economic Activity) \cite{sic-codes-2007}, which are five digit codes classifying the company's activities and the sectors they belong to. 731 sectors are organised in a hierarchy of the 21 high level sections shown in Table \ref{tab:sector-codes} in Appendix \ref{app:sector_codes}. Due to high granularity and class imbalance, detailed industry sectors were discarded and only the top-level sectors were kept. Even top-level sectors show class imbalance, as shown in Figure \ref{fig:companies-by-sector}. Fortunately this imbalance does not influence our regression modelling, since we are not building models to aiming to classify companies by their sectors. 

Raw SIC codes were cleaned, formatted, and transformed in two ways: exploding and splitting. The first transforms an entry with multiple SIC codes into many entries, each with only one code (although we also hot-encode values into new 21 columns each representing the industry section). For 2018 data, this increased the number of rows around 30\% (10,828 to 13,911). In the second approach, instead of exploding, we hot-encode the SIC codes in new 21 columns, so one row is kept for each company. This was a better approach for modelling, since exploding creates a representation bias for companies with more than one SIC code.

\paragraph{Augmenting Gender Representation}
Through iterations of modelling, we derived additional useful features from the existing data.

Although the number of employees of each gender is not provided, we can derive a number of features from the quartile divisions that reflect the relative representation of a company's workforce by gender in each quartile, as well as the relative percentages of male and female employees in total, and the skew towards one gender.

Additionally, we used bonus field data to derive information about the skew towards one gender receiving bonuses.
These fields are described in Table\ \ref{tab:additional-fields} of Appendix\ \ref{app:data-dic}. 

% \todo{update with whatever happens in the model} % Maybe reference and invite to read the modelling section as I did in the previous paragraphs.

