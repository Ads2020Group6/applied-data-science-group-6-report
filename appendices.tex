\pagebreak
\section{Description of original  and engineered features in the UK GPG data}
\label{app:data-dic}


    \begin{centering}
    \scriptsize{
        \captionof{table}{Notable fields in the UK GPG data \cite{gov-uk-gpg-data}}
    \label{tab:uk-gov-data-dic}
    \begin{tabular}{|l|l|p{10cm}|}\hline 
         & \emph{Field} & \emph{Description} \\ \hline
        
        \multirow{2}{4em}{Gender Pay Gap} 
        & \texttt{DiffMeanHourlyPercent} &  Difference between male and female mean hourly pay, as a percentage of mean male hourly pay \\ 
        & \texttt{DiffMedianHourlyPercent} & Difference between male and female median hourly pay, as a percentage of median male hourly pay \\ \hline
        \multirow{4}{4em}{Bonus-related} 
        & \texttt{DiffMeanBonusPercent } & Difference between mean male and female bonus pay, as a percentage of mean male bonus pay \\ 
        & \texttt{DiffMedianBonusPercent }  & Difference between median male and female bonus pay, as a percentage of mean male bonus pay\\ 
        & \texttt{MaleBonusPercent } &  Percentage of male employees who received any bonus\\ 
        & \texttt{FemaleBonusPercent} & Percentage of female employees who received any bonus\\ \hline
        \multirow{2}{4em}{Company Data} 
        & \texttt{EmployerSize} & One of 7 categories indicating number of employees \\ 
        & \texttt{SicCodes}  & List of Standard Industry Classifiers describing company activity \\ 
        & \texttt{CompanyLinkToGPGInfo} & Internal company GPG report for the year (optional) \\ \hline
    \end{tabular} 
    }

\end{centering}


\begin{centering}
    \scriptsize{
    \captionof{table}{Additional engineered features derived from fields in the UK GPG data \cite{gov-uk-gpg-data}}
    \label{tab:additional-fields}
    \begin{tabular}{|l|p{9cm}|}\hline 
        \emph{Field} & \emph{Description} \\ \hline
        \texttt{MalePercent} &  Percentage of men in the company \\ \hline
        \texttt{FemalePercent} & Percentage of women in the company\\ \hline
        \texttt{WorkforceGenderSkew} & Difference of \texttt{MalePercent} and \texttt{FemalePercent} \\ \hline
        \texttt{PercMaleWorkforceInTopQuartile} & Percentage of men in the top quartile compared to all men in the company\\ \hline
        \texttt{PercFemaleWorkforceInTopQuartile} & Percentage of women in the top quartile compared to all women in the company\\ \hline
        \texttt{RepresentationInTopQuartileSkew} & Difference in representation of men and women in percentage for top quartile\\ \hline
        
        \texttt{PercMaleWorkforceInUpperMiddleQuartile} & Percentage of men in the upper middle quartile compared to all men in the company\\ \hline
        \texttt{PercFemaleWorkforceInUpperMiddleQuartile} & Percentage of women in the upper middle quartile compared to all women in the company\\ \hline
        \texttt{RepresentationInUpperMiddleQuartileSkew} & Difference in representation of men and women in percentage for upper middle quartile\\ \hline
        
        \texttt{PercMaleWorkforceInLowerMiddleQuartile} & Percentage of men in the lower middle quartile compared to all men in the company\\ \hline
        \texttt{PercFemaleWorkforceInLowerMiddleQuartile} & Percentage of women in the lower middle quartile compared to all women in the company\\ \hline
        \texttt{RepresentationInLowerMiddleQuartileSkew} & Difference in representation of men and women in percentage in lower middle quartile\\ \hline
        
        \texttt{PercMaleWorkforceInLowerQuartile} & Percentage of men in the lower quartile compared to all men in the company\\ \hline
        \texttt{PercFemaleWorkforceInLowerQuartile} & Percentage of women in the lower quartile compared to all women in the company\\ \hline
        \texttt{RepresentationInLowerQuartileSkew} & Difference in representation of men and women in percentage in lower quartile\\ \hline
        
        \texttt{BonusGenderSkew} & Difference in \texttt{MaleBonusPercent} and \texttt{FemaleBonusPercent} \\ \hline
        
    \end{tabular} 
    }

\end{centering}



\section{TPOT configuration}
\label{tpot-config}

We ran TPOT with the parameters 
\code{generations=10},
\code{population\_size=150}, 
\code{cv=3} (3-fold cross-validation as in our manual model)
search,
\code{scoring='r2'} (optimising for the $R^2$ score).

The search lasted more than 5 hours, and produced models which were only slightly more accurate than the hand-tuned models.

As an example output, for the Median pay gap the best model was:

\begin{scriptsize}
\begin{verbatim}
make_pipeline(
    StackingEstimator(estimator=AdaBoostRegressor(learning_rate=0.5,
        loss="linear", n_estimators=100)),
    ExtraTreesRegressor(bootstrap=False, max_features=0.8, 
        min_samples_leaf=1, min_samples_split=7, 
        n_estimators=100)
)
\end{verbatim}
\end{scriptsize}

Unfortunately, stacked pipelines do not work with the explainability tools we used, so we did not pursue AutoML futher.

% and found the best model for the mean to be an \code{ExtraTreesRegressor} (\cite{extra-tree-regressor}) with parameters \code{bootstrap=False},
% \code{max\_features=0.7500000000000001},
% \code{min\_samples\_leaf=1},
% \code{min\_samples\_split=2},
% \code{n\_estimators=100}.

\section{SIC 2007 Sector codes}
\label{app:sector_codes}
    \begin{centering}
    \scriptsize{
        \captionof{table}{SIC 2007 Sector codes}
    \label{tab:sector-codes}
    \begin{tabular}{|c|l|}\hline 
    \emph{Code} & \emph{Description} \\ \hline 
A & Agriculture, Forestry and Fishing \\
B & Mining and Quarrying \\
C & Manufacturing \\
D & Electricity, gas, steam and air conditioning supply \\
E & Water supply, sewerage, waste management and \\
F & Construction \\
G & Wholesale and retail trade; repair of motor vehicles and \\
H & Transportation and storage \\
I & Accommodation and food service activities \\
J & Information and communication \\
K & Financial and insurance activities \\
L & Real estate activities \\
M & Professional, scientific and technical activities \\
N & Administrative and support service activities \\
O & Public administration and defence; compulsory social \\
P & Education \\
Q & Human health and social work activities \\
R & Arts, entertainment and recreation \\
S & Other service activities \\
T & Activities of households as employers; undifferentiated \\
U & Activities of extraterritorial organisations and \\
\hline
    \end{tabular} 
    }

\end{centering}